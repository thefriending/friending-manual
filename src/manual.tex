\documentclass[a4paper,11pt,titlepage]{scrartcl}

\usepackage[utf8]{inputenc}
\usepackage[T1]{fontenc} % LY1 also works
\usepackage{textcomp} % to get the right copyright, etc.
\usepackage[lining,tabular]{fbb} % so math uses tabular lining figures
\usepackage[scaled=.95,type1]{cabin} % sans serif in style of Gill Sans
\usepackage[varqu,varl]{zi4}% inconsolata typewriter
\useosf % change normal text to use proportional oldstyle figures

\usepackage{microtype}
\usepackage{graphicx}
\usepackage{enumitem}

\usepackage{listings}
\lstset{basicstyle=\ttfamily,frame=single,xleftmargin=3em,xrightmargin=3em}
\usepackage[os=win]{menukeys}
\usepackage{framed}
\usepackage{etoolbox}
\AtBeginEnvironment{leftbar}{\sffamily\small}

\usetikzlibrary{chains,arrows,shapes,positioning}
\usepackage[hidelinks]{hyperref}

%%%%
%%%%
%%%%

\newcommand\setDefaultMenuColor{\renewmenucolortheme{gray}{RGB}{70,127,215}{255,255,255}{255,255,255}}
\newcommand\setNavigationMenuColor{\renewmenucolortheme{gray}{RGB}{230,230,230}{120,144,156}{98,106,110}}
\renewmenumacro{\keys}{angularkeys}
\setDefaultMenuColor

%%%%
%%%%
%%%%

\newcommand{\textapp}[1]{{\fontfamily{cmss}\selectfont#1}}
\newcommand{\textuser}[1]{{\fontfamily{crm}\selectfont\textit{#1}}}
\newcommand{\textvar}[1]{{\textsc{#1}}}
\newcommand{\textcomp}[1]{{\fontfamily{pnc}\selectfont#1}}
\newcommand{\textui}[1]{{\fontfamily{pag}\selectfont#1}}
\newcommand{\textaction}[1]{{\fontfamily{cmtt}\selectfont#1}}
\newcommand{\textmath}[1]{{\fontfamily{zplm}\selectfont#1}}

%%%%
%%%%
%%%%

\newcommand{\Friending}{\textapp{Friending}}
\newcommand{\Screenshot}[1]{\textui{#1}}
\newcommand{\action}[1]{\textaction{#1}}
\newcommand{\abbrevation}[1]{\texttt{#1}}
\newcommand{\gterm}[2]{\item #1 - #2}

%%%%
%%%%
%%%%


%%%%
%%%%
%%%%

\newcommand\appSignUp{Sign Up}
%%%%
%%%%
%%%%

%% NEW COMMAND
%% name of screenshot, caption, label
\newcommand{\appscreenshot}[3]{
	\begin{figure}[h!]%	
	\includegraphics[width=4.4cm]{{"../assets/screenshots/#1"}.png}%
	\centering%
	\caption{#2}%
	\label{#3}%
	\end{figure}%
}
%%

\newcommand{\appbutton}[1]{
	\includegraphics[height=1em]{{"../assets/buttons/#1"}.png}%
}

\newcommand{\quicknav}{\Screenshot{Navigation Menu} (Figure~\ref{fig:navigation})}

%%%%
%%%% 
%%%%

\begin{document}

\begin{titlepage}
	\centering
	\includegraphics[width=100mm]{{"../assets/icons/friending-doc"}.png}\\
	\vspace{1.5cm}
	{\huge\bfseries Friending User Guide\par}
	\vspace{2cm}
	{\Large\itshape Jonathan Beverly (\href{https://jrbeverly.gitlab.io/}{jrbeverly})\par}
	\vfill
	{\large\textbf{Abstract}\par}
	\begin{paragraph} 
	\Friending is an online dating, friendship, and social networking website that features member-created quizzes and multiple-choice questions.  The application is available for review at \href{https://jrbeverly-friending.gitlab.io/friending/}{Friending}. 
	\end{paragraph}
	\\\vspace{0.5cm}
	{\large \today\par}
\end{titlepage}

\clearpage
\tableofcontents

\clearpage
\section{List of Figures}
\listoffigures

% Manual
%
% The beginning of the manual

\clearpage
\section{Introduction}

\subsection{Product overview}
\Friending{} is an online dating, friendship, and social networking mobile application that features user-created questionnaires and multiple choice questions.  \Friending{} has two primary features: joining groups to find people similar to you or registering for events happening in your local area.

You can create a group around one of your interests, then define questionnaires that can be used to match other members of the group.  You can join these groups and fill in questionnaires to be matched with members of the group.  The questionnaire answers are used to determine a mutual match based on the your and others responses.  You will receive a notification if a mutual match is found.

Events are managed by an event host that controls the event.  Each event has a questionnaire that you will fill out to join.  You can invite people to join your event, and when you are ready start the processing of pairing up people in the event.    The participants of your event will then be matched into pairs.  You will receive a notification when a match is set.

\subsection{Home Pages}

When opening \Friending{} for the first time you will be presented with \Screenshot{Start} (Figure~\ref{fig:start}).   If you have already signed in before then you will be taken to \Screenshot{Sign In} (Figure~\ref{fig:signin}):

\appscreenshot{Start}{Start}{fig:start}

\Screenshot{Start} is the opening page for \Friending{}. It allows you to learn more about \Friending{} through a carousel.  Simply \action{tap} on the \keys{Start the tour} button, and you will be taken to \Screenshot{Onboarding} (Figure~\ref{fig:onboarding}).

\appscreenshot{Onboarding}{Onboarding}{fig:onboarding}

You can \action{Swipe Left} or \action{Swipe Right} to move through the descriptions to learn more about \Friending.  When you are ready to begin, \action{Tap} on the \keys{Continue} button.  You will be taken to \Screenshot{Sign Up} (Figure~\ref{fig:signup}).

When you have successfully created your account or logged in, you will be presented with the \Screenshot{Home} (Figure~\ref{fig:home}):

\appscreenshot{HomePage (none)}{The first run home page}{fig:homestart}

When you are in the application you can use the \Screenshot{Navigation Menu} (Figure~\ref{fig:navigation}) to quickly navigate throughout the application.  \action{Tap} on the \appbutton{menu} button to open the \Screenshot{Navigation Menu}.   The \Screenshot{Navigation Menu} is accessible on top level pages such as but not limited to: \Screenshot{Home} (Figure~\ref{fig:home}), \Screenshot{Groups} (Figure~\ref{fig:home}) and \Screenshot{My Questionnaires} (Figure~\ref{fig:home}).

\appscreenshot{Navigation Menu}{Navigation Menu}{fig:navigation}

\appscreenshot{Homepage}{Home}{fig:home}

\clearpage
\section{Conventions}
\label{sec:conventions}

\subsection{User assumptions}
\label{sec:assumptions}
You, the user of \Friending{}, are assumed to know the basics of touch-enabled applications.  If not, please refer to a user’s manual of your mobile device.  \Friending{} is only accessible through an internet-enabled mobile device.

\subsection{Use Cases}
\label{sec:usecases}
A use case is.  All the manipulations in a use case are based on the assumption that \Friending{} is running, and you are signed in to your account unless otherwise specified.  For Sections ~\ref{sec:signup} and ~\ref{sec:signin}, you are assumed to not be signed in.  

\subsection{Notational conventions}
\label{sec:notational}
The text conventions below are used in this manual:

\begin{itemize}
\item \textapp{Computer Modern Sans Serif} is used for program names and file names.
\item Computer Modern Roman is used for normal text.
\item \textuser{Times New Roman Italics} is used for key users of the software.
\item \textmath{Palatino} is used for mathematical formulas.
\item \textvar{Palatino Small Capitals} is used for key name variables 
\item \textcomp{Avant Garde} is used for key components of the software.
\item \textui{Avant Garde Bold} is used for user interface components of the software.
\item \textaction{Computer Modern Typewriter} is used for physical actions taken by the user when interacting with the user interface of the software.
\end{itemize}

\subsection{Visual conventions}
\label{sec:visual}
The images below are used throughout the manual:

\begin{itemize}
\item \appbutton{editting} is used to indicate a field that can be toggled for editing.
\item \appbutton{calendar} is used to indicate a field that is editable and uses a calendar widget.
\item \appbutton{menu} is used to indicate a pull out menu.
\item is used to indicate the pages of a carousel menu.
\item \setDefaultMenuColor \keys{Sample}  is used to denote buttons of the user interface.
\item \setNavigationMenuColor \keys{Sample} is used to denote menu buttons of the user interface.
\end{itemize}
\setDefaultMenuColor

\subsection{Glossary of terms}
\label{sec:glossary}
The terms below are used through the manual:

\begin{itemize}
\gterm{\Friending{}}{The name of the application.}
\gterm{User}{The person who uses \Friending{}, addressed by “you”.}
\gterm{Member}{A user who joins a group.}
\gterm{Participant}{A user who fills out a response to a questionnaire.}

\gterm{Internet-Enabled Mobile Device}{A small computing device capable of accessible the internet.}
\gterm{One Time Password}{A password that is valid for a single sign on.}
\gterm{Approachable}{The trait of being approachable.}

\gterm{Fee}{A charge for \Friending{} services.}
\gterm{Billed}{A statement of money owed for \Friending{} services.}

\gterm{Profile}{The outline of "you", the user.}
\gterm{Event}{A one time event where multiple participants fill out questionnaires to receive matches with other participants in the event}
\gterm{Event Host}{A participant who initiates a event}
\gterm{Group}{A collection of members and questionnaires administered by a Group Administrator.}
\gterm{Group Administrator}{The administrator for a group that has questionnaires and members.}
\gterm{Questionnaire}{A set of questions to be answered by the participant.}
\gterm{Notification}{An email sent to a user.}

\gterm{Match}{A pairing of exactly two participants.}
\gterm{Mutual Match}{A match between two participants who have a compatibility score equal to or above the minimum match threshold.} 
\gterm{Minimum Match Threshold}{A percentage point at which the compatibility score between two participants is sufficient for a mutual match.}
\gterm{Compatibility Score}{The percentage compatibility between two participants.}
\gterm{Comparison Score}{The numeric value representing the similarity between two values.}
\gterm{Balanced Comparison Score}{A numeric value equalling $85\%$ of the best possible comparison score for a questionnaire.}
\gterm{Question Match Criteria}{The criteria determining the comparison score between two question answers.}

\gterm{Query}{A request for information from \Friending{}.}

\gterm{Number Pad}{A grid of numbers used for inputting numeric values.}

\gterm{Heterosexual}{Is romantic attraction, sexual attraction or sexual behavior between persons of the opposite sex or gender.}
\gterm{Bisexual}{Is romantic love or sexual attraction toward both males and females}
\gterm{Gay}{A male homosexual a male who experiences romantic love or sexual attraction to other males.}
\gterm{Lesbian}{A female homosexual a female who experiences romantic love or sexual attraction to other females.}
\end{itemize}

\subsection{Abbreviations}
\label{sec:abbreviations}
\begin{itemize}
\item G\&SD - Gender and Sexual Diversities
\item GUI - Graphic User Interface
\item SMS - Short Messaging Service
\end{itemize}

\subsection{Basic User interface and user interactions goals}
\label{sec:goals}
\Friending{} is inspired by mail-in oriented dating services that appeared in the early stages of "online" dating.  These computer dating services operate by having you fill out a paper questionnaire which will be mailed in with a nominal fee.  The questionnaires are geared towards people seeking a date.  

\Friending{} makes use of the internet, removing the slow and sometimes unreliable mail service.  You can fill out, submit and view your responses to questions faster than using postal mail.  Additionally \Friending{} includes more communal oriented features than the early concepts, such as enabling you to host your own events with questionnaires.

The basic goals with the user interface of \Friending{} is the following:
\begin{itemize}
\item Designing a template should be approachable due to simplified rating systems. 
\item Each view should have an obvious feature and method of navigation.
\item There is no need to require a time commitment; \Friending{} works in the background.
\end{itemize}

\subsection{Organization of this manual}
\label{sec:organization}
The remainder of this manual is organized based on use cases.  All Sections assume that you are already logged into \Friending{} with the exception of Sections ~\ref{sec:signup} and ~\ref{sec:signin}. This manual also contains troubleshooting in Section ~\ref{sec:troubleshootandtips} and  gives the limitations of the current version of \Friending{} in Section ~\ref{sec:limitations}.

\clearpage
\section{Account}
\label{sec:account}

\subsection{Signing up for your account}
\label{sec:signup}
To start using \Friending, create an account by opening the mobile application, \Friending. You will be presented with \Screenshot{Start} as shown in Figure~\ref{fig:start}. \action{Tap} on the \keys{Start the tour} button.

\appscreenshot{Sign Up}{Sign Up}{fig:signup}

Then, on \Screenshot{Sign Up} (Figure~\ref{fig:signup}), \action{enter} your \textvar{email}, \textvar{name} and \textvar{password}.  You should review the \textcomp{Terms of Service} and \textcomp{Privacy Policy} that define the expectations for you when using \Friending. Afterwards, \action{tap} on the \keys{Sign Up} button.

\subsection{Signing in to your account}
\label{sec:signin}
On \Screenshot{Start} (Figure~\ref{fig:signin}), tap the \keys{Sign In} button to start the sign in process.

\appscreenshot{Sign In}{Sign In}{fig:signin}

On \Screenshot{Sign In} (Figure~\ref{fig:signup}), \action{enter} your \textvar{email} and \textvar{password}.  Then, \action{tap} on the \keys{Sign In} button.  If you have forgotten your password, see section ~\ref{sec:reset}. 
\subsection{Viewing your settings}
\label{sec:settings}
To view your settings, \action{tap} on the \keys{Settings} button in the \Screenshot{Navigation Menu} (Figure~\ref{fig:navigation}). Once on \Screenshot{Settings} (Figure~\ref{fig:settings}), you will be able to review your settings. 

\appscreenshot{Settings}{Settings}{fig:settings}
\subsection{Updating your account details}
\label{sec:accountupdate}
To update your account details, \action{tap} on the \keys{Settings} button in the \Screenshot{Navigation Menu} (Figure~\ref{fig:navigation}).  \action{Tap} on the \keys{Account} button as shown in (Figure~\ref{fig:navigation}).

Once on the Settings (Figure 3.4), \action{enter} your changes to \textvar{email}, \textvar{phone number} and \textvar{password} fields. After you are satisfied with your changes, \action{tap} on the \keys{Save} button to save your changes.

\subsection{Resetting your password}
\label{sec:reset}
To reset your password, \action{tap} the \keys{Sign In} button on \Screenshot{Start} (Figure~\ref{fig:start}).

Then, \action{enter} your \textvar{email address} in the \textcomp{Email Address} field as in ~\ref{fig:reset}. Then tap on the \keys{Reset} link. An email will be sent your email address with a one-time password.  This password will allow you to sign in to your account, but will not change your password. The one-time-password has an expiration period of one (1) hour, when it expires, it will no longer allow access to your account.

You are encouraged to change your password after signing in with the one-time password.

\appscreenshot{Reset}{Reset your password}{fig:reset}
\subsection{Deleting your account}
\appscreenshot{Delete Account}{Delete account}{fig:accountdelete}
To delete your account, \action{tap} on the \keys{Settings} button on the \Screenshot{Navigation Menu} (Figure~\ref{fig:navigation}).

Then, while on \Screenshot{Settings} (Figure~\ref{fig:settings}), \action{tap} on the \keys{Delete Account} button.  You must agree to the deletion of your account in the \Screenshot{Confirmation Popup} (Figure~\ref{fig:accountdelete}). Your account will be deactivated and you will not receive any further matches or notifications during this period.  An email will be sent to notify you of your account deactivation and impending deletion.  Within 14 days your account will be permanently deleted along with all associated information.  If you sign in to your account within 14 days, your deletion request will be cancelled.  There is no way to recover a deleted account.

\clearpage
\section{Billing}

\subsection{Viewing your billing information}
\label{sec:viewbilling}
\appscreenshot{Billing}{Billing}{fig:billing}
To view your account billing information, \action{tap} on the \keys{Settings} button on the \quicknav.
Then, while on \Screenshot{Settings} (Figure~\ref{fig:settings}), \action{tap} on \keys{Billing} button, you will be presented with your current billing information (Figure~\ref{fig:billing}).   To download a copy of an invoice, \action{tap} on the \appbutton{invoice} button.  This will download a copy of an invoice to your device.  If any billing information is incorrect, see Section ~\ref{sec:troubleshoot} on troubleshooting.

\subsection{Update your billing information}
\appscreenshot{Credit Information}{Credit Information}{fig:credit}
To update account billing information, \action{tap} on the \keys{Settings} button on the \quicknav.  Then, while on \Screenshot{Settings} (Figure~\ref{fig:settings}), \action{tap} on the \keys{Billing} button, you will be presented with your current billing information (Figure~\ref{fig:billing}). 
 
Then, while on \Screenshot{Billing} (Figure~\ref{fig:billing}), \action{tap} on the \appbutton{creditcard} button, you will be presented with your current credit card (Figure~\ref{fig:credit}).  \action{Enter} your changes to \textvar{First Name}, \textvar{Last Name}, \textvar{Card Number}, \textvar{Zip Code} and \textvar{Expiration Date} fields.  After you are satisfied with your changes, \action{tap} on the \keys{Save} button to save your changes.  

You will be emailed a confirmation that your billing information has been changed.  

\subsection{Downloading your invoices}
\appscreenshot{Billing Notifcation Pop}{Billing Notifcation}{fig:billingnotify}

\subsection{Paying for your account services}
Paying for your account happens automatically.   You will need to set your billing information to use \Friending{}.   If \Friending{} is unable to bill your account, you will be unable to use \Friending{} until your billing information is corrected.  You will be billed on a monthly basis for your account.

You can see your billing information as specified in section ~\ref{sec:viewbilling}.

\clearpage
\section{Profile}

\subsection{Updating your profile}
\appscreenshot{Profile}{Profile}{fig:profile}
To update your personal profile information, \action{tap} on the \keys{Profile} button on the \quicknav. Then, while \Screenshot{Profile} (Figure~\ref{fig:profile}), \action{enter} your changes to \textvar{name}, \textvar{birthday}, \textvar{gender}, \textvar{orientation} and \textvar{description} fields.  You can tap on the \textcomp{Profile Image} to set your profile picture.  After you are satisfied with your changes, \action{tap} on the \keys{Save} button to save your changes.  
\\\\
The \textcomp{gender} field is limited to a binary choice of male and female.  The \textcomp{orientation} field is limited to Heterosexual, Bisexual, Homosexual.  See section ~\ref{sec:limitations} on the limitations of \Friending{} with respect to G\&SD.

\subsection{Setting your notification preferences}
\appscreenshot{Preferences}{Billing Notifcation}{fig:preferences}
To view your notification preferences, \action{tap} on the \keys{Settings} button in the \quicknav. Then, while on \Screenshot{Settings} (Figure~\ref{fig:settings}),  \action{tap} on the \keys{Notifications} button, you will be presented with your notification settings.  On \Screenshot{Preferences} (Figure~\ref{fig:preferences}) you can see the types of notifications that \Friending{} will send. To toggle email notifications for a notification type, \action{tap} on the check box.  If email notifications is enabled, the email icon will appear filled.  If email notifications is disabled, the email icon will appear empty. After you are satisfied with your changes, \action{tap} on the \keys{Save} button to save your changes.  Email notifications cannot be disabled for billing statements and account notifications.

\clearpage
\section{Groups}
\label{sec:groups}
\appscreenshot{Fill}{Billing Notifcation}{fig:fill}
\appscreenshot{Filling}{Billing Notifcation}{fig:filling}
\appscreenshot{Group Edit}{Billing Notifcation}{fig:groupedit}
\appscreenshot{Group Join}{Billing Notifcation}{fig:groupjoin}
\appscreenshot{Group}{Billing Notifcation}{fig:group}
\appscreenshot{Groups}{Billing Notifcation}{fig:groups}
\appscreenshot{Answer}{Billing Notifcation}{fig:answer}
\appscreenshot{Answers}{Billing Notifcation}{fig:answers}
\subsection{Viewing groups}
\label{sec:group-view}

\subsection{Searching groups}
\label{sec:group-search}

\subsection{Joining a group}
\label{sec:group-join}
To join a group, \action{tap} on the \keys{Group} button on the \quicknav.  Then, while on the \Screenshot{Groups} (Figure~\ref{fig:groups}), \action{enter} a \textvar{search query} into the search bar, you will be presented with groups matching your query.  Select the \textcomp{group} that matches your query to view \Screenshot{Group Details} (Figure~\ref{fig:group}).

Then, while on the \Screenshot{Group Details} page, \action{tap} on the \keys{Join} button.  The group will be added to your \Screenshot{Home Page} (Figure~\ref{fig:home}).   

\subsection{Viewing a group}
\label{sec:group-view}
To view a group, \action{tap} on the \textcomp{group} that you wish to view on the \Screenshot{Home Page} (Figure~\ref{fig:home}). This opens the \Screenshot{Group Details} (Figure~\ref{fig:group}). Then, while on the \Screenshot{Group Details}, \action{tap} on the \keys{View} button.  You will be presented with your \Screenshot{Group Home} (Figure~\ref{fig:home}).

\subsection{Leaving a group}
\label{sec:group-leave}
To leave a group, \action{tap} on the community that you wish to leave on \Screenshot{Home} (Figure~\ref{fig:home}). This opens the \Screenshot{Group Details} (Figure~\ref{fig:group}).  Then, while on \Screenshot{Group Details}, \action{tap} on the \keys{Leave} button.  The group will be removed from your \Screenshot{Home} (Figure~\ref{fig:home}).  If you wish to re-join the group, see section \ref{sec:group-join}.

\clearpage
\section{Events}
\label{sec:events}
\appscreenshot{Event Edit}{Billing Notifcation}{fig:eventedit}
\appscreenshot{Event Fill}{Billing Notifcation}{fig:eventfill}
\appscreenshot{Event Filled In}{Billing Notifcation}{fig:eventfilled}
\appscreenshot{Event Filling}{Billing Notifcation}{fig:eventfilling}
\appscreenshot{Event Joined}{Billing Notifcation}{fig:eventjoined}
\appscreenshot{Event}{Billing Notifcation}{fig:event}
\appscreenshot{Events}{Billing Notifcation}{fig:events}

\subsection{Joining an event}
\label{sec:event-join}

\subsection{Waiting to join an event}
\label{sec:event-wait}

\subsection{Event Join Requirements}
\label{sec:event-req}

\subsection{Viewing events}
\label{sec:event-view}

\subsection{Starting matching for events}
\label{sec:event-starting}

\clearpage
\section{Group}

\subsection{Creating your group}
\label{sec:group-create}

%\subsection{Viewing your group}
%\label{sec:group-view}

\subsection{Updating your group}
\label{sec:group-update}

\subsection{Viewing your group questionnaires}
\label{sec:group-questionnaire}

\subsection{Updating your group questionnaires}
\label{sec:group-questionnaire-edit}

\clearpage
\section{Questionnaires}
\label{sec:questionnaires}
\appscreenshot{Question Matching 1}{Billing Notifcation}{fig:questionmatch1}
\appscreenshot{Question Matching 2}{Billing Notifcation}{fig:questionmatch2}
\appscreenshot{Question-Editing}{Billing Notifcation}{fig:questionedit}
\appscreenshot{Questionnaire-Details}{Billing Notifcation}{fig:questiondetails}
\appscreenshot{Questionnaire-Questions}{Billing Notifcation}{fig:questions}
\appscreenshot{Questionnaires}{Billing Notifcation}{fig:questionnaires}
\appscreenshot{Question-Weighting}{Billing Notifcation}{fig:weighting}
\appscreenshot{Categories}{Billing Notifcation}{fig:categories}
\subsection{Filling out a questionnaire}
\label{sec:questionnaire-fill}

\subsection{Flagging important questions}
\label{sec:questionnaire-importance}

\subsection{Deleting a draft questionnaire}
\label{sec:questionnaire-delete}

\subsection{Viewing your submitted questionnaire}
\label{sec:submitted-view}

\subsection{Updating your submitted questionnaire}
\label{sec:submitted-update}

\subsection{Removing your submitted questionnaire}
\label{sec:submitted-delete}

\subsection{Viewing your match notifications}
\label{sec:matches}
\appscreenshot{Match}{Billing Notifcation}{fig:match}
\appscreenshot{Notifications}{Billing Notifcation}{fig:notifications}

\clearpage
\section{Questionnaire Designs}

\subsection{Creating a questionnaire}
\label{sec:questionnaire-create}

\subsection{Sending invitiations}

\clearpage
\section{Questionnaire Questions}
\label{sec:qquestions}

\subsection{Designing a questionnaire}

\subsection{Minimum score for a match}

\subsection{Category Weighting}

\subsection{Adding questions to a questionnaire}


\clearpage
\section{Questions}
\label{sec:questions}

\subsection{Question importance}
\label{sec:qimportance}

\subsection{Question category}
\label{sec:qcategory}

\subsection{Question responses}
\label{sec:qresponses}

\subsection{Ordering questions}
\label{sec:qordering}

\subsection{Question match criteria}
\label{sec:matchcriterua}

\clearpage
\section{Mutual Match}
\label{sec:matching}

\subsection{Group Matching}
\label{sec:groupmatch}
\Friending{} makes use a standard solution to the Stable Marriage Problem.

\subsection{Event Matching}
\label{sec:eventmatch}
\Friending{} makes use a standard solution to the Stable Roommate Problem.

Events are ...

Mixers are ..

\clearpage
\section{Troubleshooting \& Tips}
\label{sec:troubleshootandtips}

\subsection{Troubleshooting}
\label{sec:troubleshoot}
If you encounter difficulty while using \Friending{}, you should attempt each of the following steps until the problem is solved.

\begin{itemize}
\item Restart \Friending{}
\begin{enumerate}
\item \action{Close} the application
\item Wait a short period  after closing
\item \action{Open} \Friending{}
\end{enumerate}
\item Sign out then Sign In
\begin{enumerate}
\setNavigationMenuColor
\item \action{Tap} on the \appbutton{menu} button in the top left.
\item \action{Tap} on the \keys{Sign Out} button.
\item You will now be directed to \Screenshot{Sign In}
\item Follow Section ~\ref{sec:eventmatch} and sign in to \Friending{}
\setDefaultMenuColor
\end{enumerate}
\end{itemize}
\subsection{Tips}
\label{sec:tips}
\begin{enumerate}
\item Create a questionnaire before creating an event.
\item Group similar types of questions into categories.
\item Prototype a questionnaire by creating an event before publishing.
\end{enumerate}

If billing information on your account is incorrect, then you can contact \textbf{{billing@friending.no}}.

\clearpage
\section{Limitations}
\label{sec:limitations}

\Friending{} has limitations with respect to \abbrevation{G\&SD}.  Section ~\ref{sec:eventmatch} specifies the creation of events, specifically mixers, that ensure a match for all participants upon completion.  A mixer currently supports Heterosexual, Bisexual, Gay, and Lesbian matches.  With limited match support, the current version of \Friending{} does not support all \abbrevation{G\&SD}.  A future version may support matches providing more inclusivity for \abbrevation{G\&SD} individuals.

The following are limitations of \Friending{}:
\begin{enumerate}
\item You can only invite to an event by email address
\item You can only invite to a group by email address
\end{enumerate}

\end{document}